\documentclass{book}
\usepackage{ifxetex,ifluatex}
\ifxetex
  \usepackage{polyglossia}
  \setmainlanguage{english}
  \usepackage{fontspec}
\else\ifluatex
  \usepackage{polyglossia}
  \setmainlanguage{english}
  \usepackage{fontspec}
\else
  \usepackage[english]{babel}
  \usepackage[utf8]{inputenc}
  \usepackage[T1]{fontenc}
  \usepackage{lmodern}
\fi\fi
\usepackage{booktabs}
\usepackage[
  hashEnumerators,
  definitionLists,
  footnotes,
  inlineFootnotes,
  smartEllipses,
  fencedCode,
  contentBlocks,
  pipeTables,
  tableCaptions,
  taskLists,
]{markdown}
\begin{document}
% Typeset the document `example.md` by letting the Markdown package handle
% the conversion internally.
\markdownInput{./example.md}

% Typeset the document `example.tex` that we prepared separately using the
% Lua command-line interface of the Markdown package and that contains a
% plain TeX representation of the document `example.md`.
\InputIfFileExists{./example.tex}{}{}

\begin{markdown}
Here are some non-ASCII characters: *ěščřžýáíé*.
\end{markdown}
\end{document}
